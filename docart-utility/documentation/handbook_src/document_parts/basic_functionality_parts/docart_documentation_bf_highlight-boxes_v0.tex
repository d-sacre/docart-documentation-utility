%%%%%%%%%%%%%%%%%%%%%%%%%%%%%%%%%%%%%%%%%%%%%%%%%%%%%%%%%%%%%%%%%%%%%%%%%%%%%%%%%%%%%%
%%%%%%%%%%%%%%%%%%%%%%%%%%%%%%%%%%%%%%%%%%%%%%%%%%%%%%%%%%%%%%%%%%%%%%%%%%%%%%%%%%%%%%
%%%%%%%%%%%%%%%%%%%%%%%%%%%%%%%%%%%%%%%%%%%%%%%%%%%%%%%%%%%%%%%%%%%%%%%%%%%%%%%%%%%%%%
%% docART Utility - A Python/Lua(LaTeX) based tool for semi-automated documentation %%
%% Source: https://github.com/d-sacre/docart-documentation-utility/                 %%
%% Version: alpha-2022-04-30                                                        %%
%% License: GNU General Public License (GPLv3)                                      %%
%% Copyright (C) 2022 Martin Stimpfl, Daniel Sacré                                  %%
%%                                                                                  %%
%% This program is free software: you can redistribute it and/or modify             %%
%% it under the terms of the GNU General Public License as published by             %%
%% the Free Software Foundation, either version 3 of the License, or                %%
%% (at your option) any later version.                                              %%
%%                                                                                  %%
%% This program is distributed in the hope that it will be useful,                  %%
%% but WITHOUT ANY WARRANTY; without even the implied warranty of                   %%
%% MERCHANTABILITY or FITNESS FOR A PARTICULAR PURPOSE.  See the                    %%
%% GNU General Public License for more details.                                     %%
%%                                                                                  %%
%% You should have received a copy of the GNU General Public License                %%
%% along with this program.  If not, see <https://www.gnu.org/licenses/>.           %%
%%%%%%%%%%%%%%%%%%%%%%%%%%%%%%%%%%%%%%%%%%%%%%%%%%%%%%%%%%%%%%%%%%%%%%%%%%%%%%%%%%%%%%
%%%%%%%%%%%%%%%%%%%%%%%%%%%%%%%%%%%%%%%%%%%%%%%%%%%%%%%%%%%%%%%%%%%%%%%%%%%%%%%%%%%%%%
%%%%%%%%%%%%%%%%%%%%%%%%%%%%%%%%%%%%%%%%%%%%%%%%%%%%%%%%%%%%%%%%%%%%%%%%%%%%%%%%%%%%%%

%+++Complete+++
\section{Highlight Boxes}
	Sometimes it is necessary to highlight the importance of information. 
	Changing the font type to bold or italics is not the best way. On the 
	one hand, it breaks the reading flow and on the other hand font changes 
	can be too subtle if the reader is just scanning the document very 
	rapidly.
	%+++warningBoxExample+++
	% Generating a warning box.
	\begin{daWarningBox}
		This is a box designed to indicate a warning. Its design is by 
		purpose very minimalistic, but can be altered to suit any needs. 
		Additionally, it does not use the complete line width to provide a 
		clear visual separation between continuous text and the boxes.
	\end{daWarningBox}
	%+++warningBoxExample+++
	These boxes should only contain very little text and used sparsely; 
	otherwise the highlighting effect wears off. By design, these boxes do 
	not provide pagebreak functionality, which is intentional to force the 
	user to fill these boxes with only the necessary content.
	\newline The highlight box above was created using the following \LaTeX~
	code:
	\lstset{style=LaTeX}
	\daListingCaptionBelow{./document_parts/basic_functionality_parts/docart_documentation_bf_highlight-boxes_v0.tex}{warningBoxExample}{lst:bf:highlight-boxes:warningBox}{%
		The \LaTeX~code required for generating a warning box. %
	}
	
	In \productVersion, \productName~provides a second custom environment 
	called \daListingInline{daInfoBox}. Replacing \daListingInline{daWarningBox} with \daListingInline{daInfoBox} in the code snippet 
	above, leads to the following output:
	\begin{daInfoBox}
		This is a box designed to indicate important information. Its design
		is by purpose very minimalistic, but can be altered to suit any 
		needs. Additionally, it does not use the complete line width to 
		provide a clear visual separation between \mbox{continuous} text 
		and the boxes.
	\end{daInfoBox}
%+++Complete+++