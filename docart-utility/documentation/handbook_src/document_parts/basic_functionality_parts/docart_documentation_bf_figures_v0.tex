%%%%%%%%%%%%%%%%%%%%%%%%%%%%%%%%%%%%%%%%%%%%%%%%%%%%%%%%%%%%%%%%%%%%%%%%%%%%%%%%%%%%%%
%%%%%%%%%%%%%%%%%%%%%%%%%%%%%%%%%%%%%%%%%%%%%%%%%%%%%%%%%%%%%%%%%%%%%%%%%%%%%%%%%%%%%%
%%%%%%%%%%%%%%%%%%%%%%%%%%%%%%%%%%%%%%%%%%%%%%%%%%%%%%%%%%%%%%%%%%%%%%%%%%%%%%%%%%%%%%
%% docART Utility - A Python/Lua(LaTeX) based tool for semi-automated documentation %%
%% Source: https://github.com/d-sacre/docart-documentation-utility/                 %%
%% Version: alpha-2022-04-30                                                        %%
%% License: GNU General Public License (GPLv3)                                      %%
%% Copyright (C) 2022 Martin Stimpfl, Daniel Sacré                                  %%
%%                                                                                  %%
%% This program is free software: you can redistribute it and/or modify             %%
%% it under the terms of the GNU General Public License as published by             %%
%% the Free Software Foundation, either version 3 of the License, or                %%
%% (at your option) any later version.                                              %%
%%                                                                                  %%
%% This program is distributed in the hope that it will be useful,                  %%
%% but WITHOUT ANY WARRANTY; without even the implied warranty of                   %%
%% MERCHANTABILITY or FITNESS FOR A PARTICULAR PURPOSE.  See the                    %%
%% GNU General Public License for more details.                                     %%
%%                                                                                  %%
%% You should have received a copy of the GNU General Public License                %%
%% along with this program.  If not, see <https://www.gnu.org/licenses/>.           %%
%%%%%%%%%%%%%%%%%%%%%%%%%%%%%%%%%%%%%%%%%%%%%%%%%%%%%%%%%%%%%%%%%%%%%%%%%%%%%%%%%%%%%%
%%%%%%%%%%%%%%%%%%%%%%%%%%%%%%%%%%%%%%%%%%%%%%%%%%%%%%%%%%%%%%%%%%%%%%%%%%%%%%%%%%%%%%
%%%%%%%%%%%%%%%%%%%%%%%%%%%%%%%%%%%%%%%%%%%%%%%%%%%%%%%%%%%%%%%%%%%%%%%%%%%%%%%%%%%%%%

\section{Figures}
	In \productName, the minimum requirement for creating a figure is to specify the path to the image file(s) (supported formats: png, jpeg, pdf, eps (not recommended by the authors)); the correct layouting/formatting is handled automatically. For example, the single image with plain formatting
	%+++figExamplePlain+++
	\daFigureDefaultCaptionOff{./pictures/testpicture.png;}
	%+++figExamplePlain+++
	\newline was created by calling
	\lstset{style=LaTeX}
	\daListingCaptionOff{./document_parts/basic_functionality_parts/docart_documentation_bf_figures_v0.tex}{figExamplePlain}
	In \productVersion, this macro call creates a plain figure without a caption and a default image width of \mbox{\lstinline$0.495\\linewidth$}; changing the width or specifying the height is not possible in this version of the \productName~Utility.
	\newline To generate a figure with a caption below, replace the \lstinline$\daFigureDefaultCaptionOff$ with \lstinline$\daFigureDefaultCaptionBelow$ and add the required information for the caption,
	\lstset{style=LaTeX}
	\daListingCaptionOff{./document_parts/basic_functionality_parts/docart_documentation_bf_figures_v0.tex}{figExampleCaption}
	%+++figExampleCaption+++
	\daFigureDefaultCaptionBelow{./pictures/testpicture.png;}{%
		fig:bf:figures:example:one-image:no-grid-option:caption-below%
	}{%
		Same figure with a caption below.%
	}
	%+++figExampleCaption+++
	which generates the \mbox{figure \ref{fig:bf:figures:example:one-image:no-grid-option:caption-below}}. The general syntax for figures with a single image and captions is
	\lstinline$\daFigureDefaultCaptionBelow{FILEPATH;}{LABEL}{CAPTION}$. For a detailed explanation what the \lstinline$LABEL$ property can be used for, please refer to \mbox{section \ref{sec:bf:labels}}.
	
	\newpage To load multiple images with one shared caption, simply specify all of the file paths separated by a semicolon: \lstinline$\daFigureDefaultCaptionBelow{FILEPATH1;FILEPATH2;}{LABEL}{CAPTION}$
	\newline By default, the images will be arranged into two columns and scaled to \lstinline$0.495\linewidth$. 
	%+++figExampleTwoImg+++
	\daFigureDefaultCaptionBelow{
		./pictures/testpicture.png;
		./pictures/testpicture.png;
	}{fig:bf:figures:example:two-images:no-grid-option:caption-below}{%
		Loading two images with a caption below.%
	}
	%+++figExampleTwoImg+++
	\newline The \mbox{figure \ref{fig:bf:figures:example:two-images:no-grid-option:caption-below}} was generated by the following code:
	\lstset{style=LaTeX}
	\daListingCaptionOff{./document_parts/basic_functionality_parts/docart_documentation_bf_figures_v0.tex}{figExampleTwoImg}
	For better readability, the lines may be broken after the semicolon.
	\begin{daWarningBox}
		In contrast to the recommended \LaTeX~convention to prevent issues occurring by line breaks in macro arguments by terminating the lines with a comment \lstinline$\%$, this has to be omitted for the (first) mandatory argument as this would lead to parsing issues and an aborted compilation.
	\end{daWarningBox}
	A layout with more than two columns is is possible by specifying the \lstinline$grid$ option (see \mbox{figure \ref{fig:bf:figures:example:usage-grid-option-and-over-specify}}).
	\vspace{-0.125cm}
	%+++figExampleThreeCols+++
	\daFigureDefaultCaptionBelow[grid=2x3]{
		./pictures/testpicture.png;
		./pictures/testpicture.png;
		./pictures/testpicture.png;
		./pictures/testpicture.png;
		./pictures/testpicture.png;
		./pictures/testpicture.png;
		./pictures/testpicture.png;
	}{fig:bf:figures:example:usage-grid-option-and-over-specify}{%
		Example how to use the \lstinline$grid=2x3$ option to force a three column layout with two rows total. Specifying more images (seven instead of six) results in the excess files being ignored.%
	}
	%+++figExampleThreeCols+++

	\newpage
	The \mbox{figure \ref{fig:bf:figures:example:usage-grid-option-and-over-specify}} was created by the command shown in \mbox{listing \ref{lst:bf:figures:examples:grid-option}}.
	\lstset{style=LaTeX}
	\daListingCaptionBelow{./document_parts/basic_functionality_parts/docart_documentation_bf_figures_v0.tex}{figExampleThreeCols}{lst:bf:figures:examples:grid-option}{%
		 The command used to generate \mbox{figure \ref{fig:bf:figures:example:usage-grid-option-and-over-specify}}.%
	}
	The option \lstinline$grid=NxM$ creates a figure layout with \lstinline$N$ rows and \lstinline$M$ columns (\mbox{e.\,g.} \mbox{figure \ref{fig:bf:figures:example:usage-grid-option-and-over-specify}}: two rows, three columns). Additionally, \mbox{listing \ref{lst:bf:figures:examples:grid-option}} shows the way how more specified file paths (seven in the example) as available image slots (maximum amount of images: \lstinline$ROWS$ times \lstinline$COLS$, \mbox{e.\,g.} \mbox{figure \ref{fig:bf:figures:example:usage-grid-option-and-over-specify}}: {\lstinline$2$$\cdot$\lstinline$3$$\,=\,$\lstinline$6$}) are handled. Only the amount of images that can be typeset are processed, starting from the top of the list to the bottom, ignoring superfluous file paths. A similar procedure is applied for the case of missing file paths (see \mbox{figure \ref{fig:bf:figures:example:grid:even:under-specifying}} and \mbox{figure \ref{fig:bf:figures:example:grid:odd:under-specifying}}).\\[-0.325cm]
	\daFigureDefaultCaptionBelow[grid=2x4]{
		./pictures/testpicture.png;
		./pictures/testpicture.png;
		./pictures/testpicture.png;
		./pictures/testpicture.png;
		./pictures/testpicture.png;
		./pictures/testpicture.png;
		./pictures/testpicture.png;
	}{fig:bf:figures:example:grid:even:under-specifying}{%
		Example how specifying less images (seven instead of eight) than would fit in a grid with even number of columns (2x4 grid) is handled.%
	}
	\vspace{-0.4cm}
	\daFigureDefaultCaptionBelow[grid=2x3]{
		./pictures/testpicture.png;
		./pictures/testpicture.png;
		./pictures/testpicture.png;
		./pictures/testpicture.png;
		./pictures/testpicture.png;
	}{fig:bf:figures:example:grid:odd:under-specifying}{%
		Example how specifying less images (five instead of six) than would fit in a grid with odd number of columns (2x3 grid) is handled.%
	}

	\newpage
	The cases depicted in \mbox{figure \ref{fig:bf:figures:example:grid:even:under-specifying}} and \mbox{figure \ref{fig:bf:figures:example:grid:odd:under-specifying}} showcase the behavior of the \productName~Utility when too few file paths are specified for a layout with even/odd number of columns, respectively. All the specified images will be typeset; instead of setting a placeholder image, the first row will contain one element less than the other rows. For a better visual impression, the first row will also be centered differently.
	\newline In contrast to the single and two column layouts shown in  \mbox{figure \ref{fig:bf:figures:example:one-image:no-grid-option:caption-below}} and \mbox{figure \ref{fig:bf:figures:example:two-images:no-grid-option:caption-below}}, respectively, the image width in the three and four column layouts of \mbox{figure \ref{fig:bf:figures:example:usage-grid-option-and-over-specify}},  \mbox{figure \ref{fig:bf:figures:example:grid:even:under-specifying}} and \mbox{figure \ref{fig:bf:figures:example:grid:odd:under-specifying}} is reduced to accommodate the additional images without exceeding the type area. Although it is technically possible, the authors would not recommend to use more than four columns, since otherwise the images will get too small. Similar to the columns, the number of rows has no technical limit except the page height. Due to the fact that the height of the individual images is not adjustable and the visual pollution if too many images/much content are/is on a single page, the authors recommend not to use more than six rows (depending on the height of the images).  
	\begin{daInfoBox}
		In \LaTeX, a single figure object cannot span multiple pages. Please take this into account when designing the individual images and while you are creating your document structure. If necessary, split the images into two or more independent figure objects. 
	\end{daInfoBox}