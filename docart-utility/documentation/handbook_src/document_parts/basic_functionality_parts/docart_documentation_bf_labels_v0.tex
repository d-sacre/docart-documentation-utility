%%%%%%%%%%%%%%%%%%%%%%%%%%%%%%%%%%%%%%%%%%%%%%%%%%%%%%%%%%%%%%%%%%%%%%%%%%%%%%%%%%%%%%
%%%%%%%%%%%%%%%%%%%%%%%%%%%%%%%%%%%%%%%%%%%%%%%%%%%%%%%%%%%%%%%%%%%%%%%%%%%%%%%%%%%%%%
%%%%%%%%%%%%%%%%%%%%%%%%%%%%%%%%%%%%%%%%%%%%%%%%%%%%%%%%%%%%%%%%%%%%%%%%%%%%%%%%%%%%%%
%% docART Utility - A Python/Lua(LaTeX) based tool for semi-automated documentation %%
%% Source: https://github.com/d-sacre/docart-documentation-utility/                 %%
%% Version: alpha-2022-04-30                                                        %%
%% License: GNU General Public License (GPLv3)                                      %%
%% Copyright (C) 2022 Martin Stimpfl, Daniel Sacré                                  %%
%%                                                                                  %%
%% This program is free software: you can redistribute it and/or modify             %%
%% it under the terms of the GNU General Public License as published by             %%
%% the Free Software Foundation, either version 3 of the License, or                %%
%% (at your option) any later version.                                              %%
%%                                                                                  %%
%% This program is distributed in the hope that it will be useful,                  %%
%% but WITHOUT ANY WARRANTY; without even the implied warranty of                   %%
%% MERCHANTABILITY or FITNESS FOR A PARTICULAR PURPOSE.  See the                    %%
%% GNU General Public License for more details.                                     %%
%%                                                                                  %%
%% You should have received a copy of the GNU General Public License                %%
%% along with this program.  If not, see <https://www.gnu.org/licenses/>.           %%
%%%%%%%%%%%%%%%%%%%%%%%%%%%%%%%%%%%%%%%%%%%%%%%%%%%%%%%%%%%%%%%%%%%%%%%%%%%%%%%%%%%%%%
%%%%%%%%%%%%%%%%%%%%%%%%%%%%%%%%%%%%%%%%%%%%%%%%%%%%%%%%%%%%%%%%%%%%%%%%%%%%%%%%%%%%%%
%%%%%%%%%%%%%%%%%%%%%%%%%%%%%%%%%%%%%%%%%%%%%%%%%%%%%%%%%%%%%%%%%%%%%%%%%%%%%%%%%%%%%%

\section{References and Labels}
	\label{sec:bf:labels}
	%+++refLabelTextExample+++
	\subsection{Usage}
		\label{subsec:bf:labels:usage}
		%+++refLabelTextExample+++
		\lstset{style=LaTeX}
		%+++refLabelTextExample+++
		\productName~and \LaTeX~in general provide a very simple interface for setting labels and referencing them. The interface works the same for referring to chapters, sections, etc. or objects like figures, tables, code listings, etc.. In \mbox{section \ref{subsec:bf:labels:usage}}, on 
		\mbox{page \pageref{subsec:bf:labels:usage}}, the basic functionality will be explained.
		%+++refLabelTextExample+++
		\begin{daInfoBox}
			It requires up-to two compilations until changes in references and labels take effect.
		\end{daInfoBox}
		To set a label in any (text) position of the document, one uses the \lstinline$\label{LABEL}$ command with \lstinline{LABEL} being an unique identifier which the user can choose (almost) freely.
		\newline Referencing can be done via \lstinline$\ref{LABEL}$ or \lstinline$\pageref{LABEL}$, which return the current table-of-content layer or the page the \lstinline$LABEL$ is placed. The two references at the beginning of this page were generated by the code shown in \mbox{listing \ref{lst:bf:labels:usage:example-labels-in-text}}.
		%+++listingLabelExample+++
		\daListingCaptionBelow{%
			./document_parts/basic_functionality_parts/docart_documentation_bf_labels_v0.tex%
		}{refLabelTextExample}{lst:bf:labels:usage:example-labels-in-text}{%
			An example how to refer to certain text passages of the document.%
		}
		%+++listingLabelExample+++
		For \productName~objects like figures, tables and code listings, when you choose a command variant which provides a caption, a label is automatically generated. As an example, \mbox{listing \ref{lst:bf:labels:usage:example-labels-in-text}} is generated by the code below.
		\daListingCaptionOff{./document_parts/basic_functionality_parts/docart_documentation_bf_labels_v0.tex}{listingLabelExample}
		For this example, the label \lstinline$lst:bf:labels:usage:example-labels-in-text$ is automatically created and can be referred to like described above. The only difference is that \lstinline$\ref{LABEL}$ will return the number of the object instead of the table-of-content layer. For an overview what \lstinline$\ref{LABEL}$ returns depending upon the object, please refer to \mbox{page \pageref{tab:bf:labels:naming-convention:recommended-type-ids}},  \mbox{table \ref{tab:bf:labels:naming-convention:recommended-type-ids}}.
		
	
	\newpage
	\subsection{Recommended Label Naming Convention}
		Although label names may contain any character, it is recommended to waive using \LaTeX~special or non-standard-ASCII characters like umlaute.
		\newline A suggestion from the \LaTeX~community is to use label names that contain the type, followed by a colon and a brief description of the object that is labeled:\\[-1cm]		
		\begin{center}
			\lstinline$TYPE:DESCRIPTOR$
		\end{center}
		\vspace{-0.5cm}
		The list of recommended \lstinline$TYPE$ identifiers is given in \mbox{table \ref{tab:bf:labels:naming-convention:recommended-type-ids}}.
		\daTableDefaultFancyCaptionBelow{./tables/docART_label_recommended-type-ids.csv}{tab:bf:labels:naming-convention:recommended-type-ids}{%
			Recommended \lstinline$TYPE$ identifiers for different type of labeled objects. %
		}
		For example, if one wants to refer to \mbox{table \ref{tab:bf:labels:naming-convention:recommended-type-ids}} which contains the recommended \lstinline$Type$ identifiers, a possible label would be:\\[-1cm]		
		\begin{center}
			\lstinline$tab:recommended-type-ids$
		\end{center} 
		\vspace{-0.325cm}
		Please note that instead of white spaces, dashes were used in the descriptor to eliminate any chance of parsing problems in the \LaTeX-Backend.
		\newline The authors of the \productName~Utility recommend a variant of this scheme. Especially for documentations, when multiple individuals have to work on different parts of the document, it is very helpful to provide further information encoded within the label name; especially where the label is located. For the \productName~Utility documentation, the following scheme has been used:\\[-1cm]		
		\begin{center}
			\lstinline$TYPE:TOCLAYERS:DESCRIPTOR$
		\end{center} 
		\vspace{-0.25cm}
		For example, the label for \mbox{table \ref{tab:bf:labels:naming-convention:recommended-type-ids}} looks like this:\\[-1cm]		
		\begin{center}
			\lstinline$tab:bf:labels:naming-convention:recommended-type-ids$
		\end{center}
		\vspace{-0.25cm}
		The label consists of the \lstinline$TYPE$ identifier, which is followed by colon-separated toc-layers (in this case: abbreviated names of the respective chapter, section, subsection) before being ended by the \lstinline$DESCRIPTOR$. This makes the label name a lot longer and leads to issues when toc-element names change. However, the authors' experience shows that if the outline of the document is planned before hand (what always should happen for any good documentation), this approach facilitates collaborative writing.  