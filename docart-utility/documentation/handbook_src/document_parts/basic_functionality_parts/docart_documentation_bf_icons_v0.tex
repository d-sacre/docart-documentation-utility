%%%%%%%%%%%%%%%%%%%%%%%%%%%%%%%%%%%%%%%%%%%%%%%%%%%%%%%%%%%%%%%%%%%%%%%%%%%%%%%%%%%%%%
%%%%%%%%%%%%%%%%%%%%%%%%%%%%%%%%%%%%%%%%%%%%%%%%%%%%%%%%%%%%%%%%%%%%%%%%%%%%%%%%%%%%%%
%%%%%%%%%%%%%%%%%%%%%%%%%%%%%%%%%%%%%%%%%%%%%%%%%%%%%%%%%%%%%%%%%%%%%%%%%%%%%%%%%%%%%%
%% docART Utility - A Python/Lua(LaTeX) based tool for semi-automated documentation %%
%% Source: https://github.com/d-sacre/docart-documentation-utility/                 %%
%% Version: alpha-2022-04-30                                                        %%
%% License: GNU General Public License (GPLv3)                                      %%
%% Copyright (C) 2022 Martin Stimpfl, Daniel Sacré                                  %%
%%                                                                                  %%
%% This program is free software: you can redistribute it and/or modify             %%
%% it under the terms of the GNU General Public License as published by             %%
%% the Free Software Foundation, either version 3 of the License, or                %%
%% (at your option) any later version.                                              %%
%%                                                                                  %%
%% This program is distributed in the hope that it will be useful,                  %%
%% but WITHOUT ANY WARRANTY; without even the implied warranty of                   %%
%% MERCHANTABILITY or FITNESS FOR A PARTICULAR PURPOSE.  See the                    %%
%% GNU General Public License for more details.                                     %%
%%                                                                                  %%
%% You should have received a copy of the GNU General Public License                %%
%% along with this program.  If not, see <https://www.gnu.org/licenses/>.           %%
%%%%%%%%%%%%%%%%%%%%%%%%%%%%%%%%%%%%%%%%%%%%%%%%%%%%%%%%%%%%%%%%%%%%%%%%%%%%%%%%%%%%%%
%%%%%%%%%%%%%%%%%%%%%%%%%%%%%%%%%%%%%%%%%%%%%%%%%%%%%%%%%%%%%%%%%%%%%%%%%%%%%%%%%%%%%%
%%%%%%%%%%%%%%%%%%%%%%%%%%%%%%%%%%%%%%%%%%%%%%%%%%%%%%%%%%%%%%%%%%%%%%%%%%%%%%%%%%%%%%

%+++Complete+++
%+++bficons+++
\section{Icons (\daTestIcon) in Text}
%+++bficons+++
	In comparison to many other word processors, the \productName~Backend allows inclusion of custom icons into the continuous text. It also works for headings, tables, captions and lists. The icons can be either images (png, jpg, pdf), TikZ-Elements and more. There is also a native scalability with font size, as it is demonstrated below with some of the default icons the \productName~Utility ships with. %\\[0.5cm] 
	\begin{daInfoBox}
		For version \productVersion, the authors do not recommend the user to create custom icons and therefore do not provide a how-to-guide.
		The reason being that in one of the next releases, substantial changes will be made to the icons Backend, which should make the icon creation by the
		end user way easier and more consistent.
	\end{daInfoBox}
	%+++bficons+++
	This is a test normal sized text with an \daTestIcon icon within the text.\\[0.125cm]
	This is is a normal sized text with an icon at the end \daTestIcon.\\[0.125cm]
	{\large This is a large text with an \daTestIcon icon within the text.}\\[0.125cm]
	{\Large This is a Large text with an \daTestIcon icon within the text.}\\[0.125cm]
	{\LARGE This is a LARGE text with an \daTestIcon icon within the text.}\\[0.125cm]
	%+++bficons+++
	
	\lstset{style=LaTeX}
	\daListingTEMPLATE{./document_parts/basic_functionality_parts/docart_documentation_bf_icons_v0.tex}{bficons}{lst:bf:icons-in-text:icon-example}{%
		\LaTeX~code for using icons in the text.
	}
	
	The \lstinline$\daTestIcon$ comes in handy if a new font/typeface should be assessed.
	\newline For everyday use,
	\productName~provides a selection of default icons. The icons can be accessed by calling the macro \lstinline[emphstyle={\color{orange}\bfseries},moreemph={icon}]$\daIcon{ICONNAME}$. For example,
	\lstinline[emphstyle={\color{orange}\bfseries},moreemph={icon}]$\daIcon{warning}$ 
	generates an exclamation mark within a yellow triangle \daIcon{warning}. Icons with names containing \enquote{\lstinline$var$} can be scaled manually. For example,  \lstinline[emphstyle={\color{orange}\bfseries},moreemph={icon}]$\daIcon[10pt]{varwrench}$ generates a wrench \daIcon[10pt]{varwrench} of width \mbox{10\,pt}. The table \ref{tab:bf:iconOverview} lists all the currently available icons in \mbox{\productName}~\mbox{\productVersion}.
	
	\daTableDefaultFancyCaptionAbove{./tables/docART_icon-overview.csv}{tab:bf:iconOverview}{Overview of available icons in \productName~\productVersion.}

%+++Complete+++
