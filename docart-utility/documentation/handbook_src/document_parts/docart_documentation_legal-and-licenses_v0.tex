%%%%%%%%%%%%%%%%%%%%%%%%%%%%%%%%%%%%%%%%%%%%%%%%%%%%%%%%%%%%%%%%%%%%%%%%%%%%%%%%%%%%%%
%%%%%%%%%%%%%%%%%%%%%%%%%%%%%%%%%%%%%%%%%%%%%%%%%%%%%%%%%%%%%%%%%%%%%%%%%%%%%%%%%%%%%%
%%%%%%%%%%%%%%%%%%%%%%%%%%%%%%%%%%%%%%%%%%%%%%%%%%%%%%%%%%%%%%%%%%%%%%%%%%%%%%%%%%%%%%
%% docART Utility - A Python/Lua(LaTeX) based tool for semi-automated documentation %%
%% Source: https://github.com/d-sacre/docart-documentation-utility/                 %%
%% Version: alpha-2022-04-30                                                        %%
%% License: GNU General Public License (GPLv3)                                      %%
%% Copyright (C) 2022 Martin Stimpfl, Daniel Sacré                                  %%
%%                                                                                  %%
%% This program is free software: you can redistribute it and/or modify             %%
%% it under the terms of the GNU General Public License as published by             %%
%% the Free Software Foundation, either version 3 of the License, or                %%
%% (at your option) any later version.                                              %%
%%                                                                                  %%
%% This program is distributed in the hope that it will be useful,                  %%
%% but WITHOUT ANY WARRANTY; without even the implied warranty of                   %%
%% MERCHANTABILITY or FITNESS FOR A PARTICULAR PURPOSE.  See the                    %%
%% GNU General Public License for more details.                                     %%
%%                                                                                  %%
%% You should have received a copy of the GNU General Public License                %%
%% along with this program.  If not, see <https://www.gnu.org/licenses/>.           %%
%%%%%%%%%%%%%%%%%%%%%%%%%%%%%%%%%%%%%%%%%%%%%%%%%%%%%%%%%%%%%%%%%%%%%%%%%%%%%%%%%%%%%%
%%%%%%%%%%%%%%%%%%%%%%%%%%%%%%%%%%%%%%%%%%%%%%%%%%%%%%%%%%%%%%%%%%%%%%%%%%%%%%%%%%%%%%
%%%%%%%%%%%%%%%%%%%%%%%%%%%%%%%%%%%%%%%%%%%%%%%%%%%%%%%%%%%%%%%%%%%%%%%%%%%%%%%%%%%%%%

\chapter{Legal Disclaimer}
	\begin{itemize}
		\item Trademarks are used throughout the document to describe software, hardware and \mbox{functionality}. None of the authors is holding one of the trademarks mentioned nor is affiliated with any of the trademark holders. All trademarks belong to their respective owners. Before reusing the trademarks in another product, the reader is responsible to check whether the intended usage is prohibited. The authors of this documentation deny any legal responsibility for damages caused by unverified re-usage.
		\item  All licenses have been verified to the best knowledge of the authors during the writing of the document. The authors cannot guarantee that the licensing status of fonts, programs or tools etc. change over time. The reader is responsible for verifying if the intended use case is permitted by the license. The authors of this documentation deny any legal responsibility for damages caused by unverified re-usage.
		\item All web hyperlinks have been verified during the writing of the document. The authors of this documentation deny any legal responsibility for correctness of the information presented on the linked web pages or damages caused by visiting the web pages.
		\item Any suggestions and advises regarding good practices, typesetting rules or maximizing efficiency are personal opinions of the authors based upon their scientific typesetting training and experience. The authors deny any responsibility if the product you try to produce with \productName~does not look good or is not being approved by (typesetting) authorities.
		\item The software is provided \enquote{as is}, without warranty of any kind, express or implied, including but not limited to the warranties of merchantability, fitness for a particular purpose and noninfringement. In no event shall the authors or copyright holders be liable for any claim, damages or other liability, whether in an action of contract, tort or otherwise, arising from, out of or in connection with the software or the use or other dealings in the software.
	\end{itemize}


\newpage	
\chapter{Licensing (Quick Overview)}
	Since the \productName~Utility is built-upon many other OpenSource projects, its Backend is comprised of products with many different licenses. The following licenses are present in the products that are essential for a working \productName~Utility:
	\begin{center}
		MIT, GNU Public License (GPL), LaTeX Project Public License (LPPL),\\
		The Python Software Foundation (PSF) License Agreement
	\end{center}
	The fonts which the \productName~Utility ships are released under MIT, Bitstream Vera License and GUST License. However, if required, the fonts can be easily swapped (see \mbox{section \ref{chap:adapting-default-theme}}).
	\newline Since many of the Backend products are bundles, their individual sub-modules might have different licenses. The agreed upon standard is that the redistribution is possible under the bundle's redistribution license. All the \productName~Utility's end user cases are in agreement with the products' individual licenses, so the usage of the \productName~Utility itself is safe (except re-bundling/-distribution). To help the end user, the authors provide a detailed listing of all the products and their respective licensing in \mbox{appendix \ref{appendix:chap:licensing}}. However, the authors do not take any responsibility that the information is correct. If in doubt, the end user is responsible to check whether the intended usage is prohibited. The Free Software Foundation’s (FSF) definition and the Debian Free Software Guidelines can be helpful for deciding edge cases.
	\newline The \productName~Utility itself is licensed under \href{https://www.gnu.org/licenses/gpl-3.0.en.html}{GNU General Public License (GPLv3)}. This includes all \LaTeX-, Python-, and Lua-files as well as the \productName~specific dcmts-files and the documentation (pdf as well as source). The only exception is the \productName~logo, which is released under \href{https://creativecommons.org/licenses/by-sa/4.0/deed.en}{Creative Commons Attribution-ShareAlike 4.0 International (CC BY-SA 4.0)}.\\
	If you are re-using some part of the \productName~Utility's source code, please give credit according to the GPLv3 rules as follows:
	\begin{center}
		Martin Stimpfl, Daniel Sacré: \productName~Utility, \productVersion,\\ 
		License: GNU General Public License (GPLv3).
	\end{center}

	\begin{daInfoBox}
		Since a \LaTeX~file is a mixture of source code and content, there is no clear definition of when giving credit/re-publishing the source code is necessary. The authors therefore decided the following: The (\LaTeX) source code written by the user contains more data than actual \productName~Utility source code. Consequently, it has to be treated as data, which is belonging to its creator, the end user. All the \mbox{\productName}~Utility does is to process the data into a formatted output as binary pdf, thus the pdf output can be distributed under any license. As long as the end user did not do substantial changes to the \productName~Utility source code, it does not have to be published. For details about what substantial source code changes are, please see the examples in \mbox{appendix \ref{appendix:sec:licensing:examples-when-credit-or-re-publishing}}, \mbox{page \pageref{appendix:sec:licensing:examples-when-credit-or-re-publishing}} and following. 
	\end{daInfoBox}

	