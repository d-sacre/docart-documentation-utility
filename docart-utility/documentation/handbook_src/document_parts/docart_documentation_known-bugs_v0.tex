%%%%%%%%%%%%%%%%%%%%%%%%%%%%%%%%%%%%%%%%%%%%%%%%%%%%%%%%%%%%%%%%%%%%%%%%%%%%%%%%%%%%%%
%%%%%%%%%%%%%%%%%%%%%%%%%%%%%%%%%%%%%%%%%%%%%%%%%%%%%%%%%%%%%%%%%%%%%%%%%%%%%%%%%%%%%%
%%%%%%%%%%%%%%%%%%%%%%%%%%%%%%%%%%%%%%%%%%%%%%%%%%%%%%%%%%%%%%%%%%%%%%%%%%%%%%%%%%%%%%
%% docART Utility - A Python/Lua(LaTeX) based tool for semi-automated documentation %%
%% Source: https://github.com/d-sacre/docart-documentation-utility/                 %%
%% Version: alpha-2022-04-30                                                        %%
%% License: GNU General Public License (GPLv3)                                      %%
%% Copyright (C) 2022 Martin Stimpfl, Daniel Sacré                                  %%
%%                                                                                  %%
%% This program is free software: you can redistribute it and/or modify             %%
%% it under the terms of the GNU General Public License as published by             %%
%% the Free Software Foundation, either version 3 of the License, or                %%
%% (at your option) any later version.                                              %%
%%                                                                                  %%
%% This program is distributed in the hope that it will be useful,                  %%
%% but WITHOUT ANY WARRANTY; without even the implied warranty of                   %%
%% MERCHANTABILITY or FITNESS FOR A PARTICULAR PURPOSE.  See the                    %%
%% GNU General Public License for more details.                                     %%
%%                                                                                  %%
%% You should have received a copy of the GNU General Public License                %%
%% along with this program.  If not, see <https://www.gnu.org/licenses/>.           %%
%%%%%%%%%%%%%%%%%%%%%%%%%%%%%%%%%%%%%%%%%%%%%%%%%%%%%%%%%%%%%%%%%%%%%%%%%%%%%%%%%%%%%%
%%%%%%%%%%%%%%%%%%%%%%%%%%%%%%%%%%%%%%%%%%%%%%%%%%%%%%%%%%%%%%%%%%%%%%%%%%%%%%%%%%%%%%
%%%%%%%%%%%%%%%%%%%%%%%%%%%%%%%%%%%%%%%%%%%%%%%%%%%%%%%%%%%%%%%%%%%%%%%%%%%%%%%%%%%%%%

\chapter{Known Bugs}
	\lstset{style=LaTeX}
	\begin{enumerate}
		\item Some pdf viewers show white lines between the rows in a code listing. This also depends on the zoom level. % Fix: Spacing setting between listing lines?
		\item \LaTeX~listings have issues that all words in text identical to keywords will be highlighted; cannot use \textbackslash~for regex/identification.
		\item Highlight boxes can currently not be made from a template and have to be coded from scratch.
		\item At least two, sometimes three compilations required until tables have correct layout/optics.
		\item Table row highlighting sometimes dependent upon tables that came before.
		\item Icon alignment is not accurate, especially in tables.
		\item \lstinline{\daListingInline} and \lstinline{\lstinline} mess up spacing over line break.
		\item \lstinline{\daListingInline} requires escaping of \LaTeX~special characters passed in as an argument.
		\item Indentation correction for listings randomly does not work properly.
		\item Last line in listings sometimes removed. Seems to be related with the usage of the \enquote{\lstinline$all$} tag. Fix: Add empty line to file. % Case: LaTeX, only } in last line; also happend with C++
		\item Empty lines in Lua scripts will be removed.
		\item Setting any expression as \lstinline$emph$ in any listing style will affect all other styles.
		\item Two sections on the same page have a way too big vertical blank space in between. Fix: \lstinline$\vspace{-1.5cm}$ before the second \lstinline{\section} call.
		\item No proper math font loaded. The default \LaTeX~math font does not fit the rest of the fonts (size, shape).
		\item The aux-files on rare occasion have to be manually deleted before the compilation (otherwise compilation will not finish successfully). Observed case: Changing command name of macro which was present in a \LaTeX~list (table of contents, section argument, etc.). Old command name still present in aux-file, which led to an aborted compilation before it could be completed (aux-files therefore did not get properly recreated).
	\end{enumerate}
